\documentclass[10pt, compress]{beamer}

\usetheme[usetitleprogressbar]{m}

\usepackage{booktabs}


\title{Couplage FEM/DGM}
\subtitle{Analyse des méthodes et proposition de couplage}
\date{Année 2014-2015}
\author{Mathieu Gaborit}
\institute{Université du Maine}

\begin{document}

\maketitle

\begin{frame}[fragile]
    \frametitle{Introduction}

    \begin{itemize}
        \item Méthodes numériques : enjeu majeur pour la simulation de systèmes complexes
        \item Grande diversité dans les méthodes disponibles
        \item Fortes spécificités pour chaque méthode
        \item Méthodes classiques : FEM, DGM, etc...
    \end{itemize}

    \pause
    \begin{center}
        \alert{\textbf{Comment combiner deux méthodes pour profiter des avantages de tous leurs avantages ?}}
    \end{center}
\end{frame}

\begin{frame}
    \frametitle{References}

    \begin{itemize}
        \item \textbf{A discontinuous Galerkin Method with Plane Waves for Sound Absorbing Materials}, \textit{Int. J.
            Numer. Engng}, G. Gabard, O.  Dazel
        \item \textbf{A comparison of wave-based discontinuous Galerkin, ultra-week and least-square method for wave
            problems}, \textit{Int. J.
            Numer. Engng}, G. Gabard, P. Gamallo, T. Huttunen
        \item \textbf{Analyse Numérique : une approche mathématique}, M. Schatzman
    \end{itemize}
\end{frame}

\end{document}
