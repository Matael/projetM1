\documentclass[a4paper, 11pt]{article}

	\usepackage[utf8]{inputenc}
\usepackage[T1]{fontenc}
\usepackage[french]{babel}
\usepackage{amsmath}
\usepackage{amssymb}
\usepackage{cancel}

% commands definitions (math)
\newcommand\dx{\mathrm{d}x}


    \title{FEM 1D -- Plane Wave excitation}
    \author{Mathieu Gaborit}


\begin{document}
	\maketitle

	We consider a 1D duct of length L along $x$.

	\paragraph{Boundary conditions}

	The duct is closed with inifite impedance wall at $x = L$ and open at $x = 0$.

	We consider a plane wave harmonic excitation at $x = 0$.

	\begin{equation}
		\left.\frac{\partial p}{\partial n}\right\|_{x=L} = 0 \label{FEM1DPW:BC_L}
	\end{equation}

	\begin{equation}
		\left.p\right\|_{x=0} = p_i + p_rA \label{FEM1DPW:BC_0}
	\end{equation}


	\paragraph{Wave equation}

	One can consider the general wave equation without source:

	\begin{equation}
		\Delta p + \frac{1}{c^2} \frac{\partial^2 p}{\partial t^2} = 0 \label{FEM1DPW:wave_eq}
	\end{equation}

	Assuming a 1D harmonic problem, equation~\eqref{FEM1DPW:wave_eq} reads as the Helmholtz equation~\eqref{FEM1DPW:helmholtz}.

	\begin{equation}
		\Delta p + k^2 p  = 0 \Rightarrow \frac{\partial^2p}{\partial x^2} + k^2p = 0\label{FEM1DPW:helmholtz}
	\end{equation}

	\paragraph{Variational Formulation}

	Multiplying equation~\eqref{FEM1DPW:helmholtz} by a test field and integrating over the domain, one transfer the
	problem under variational formalism :

	\begin{eqnarray*}
		\int_0^L\frac{\partial^2p}{\partial x^2}v\dx + k^2\int_0^L pv\dx = 0 ,&& \forall v \\
		\underbrace{\bigg[~p'v~\bigg]_0^L}_{(*)}-\int_0^Lp'v'\dx + k^2\int_0^Lpv\dx = 0, && \forall v
	\end{eqnarray*}

	Using equation~\eqref{FEM1DPW:BC_L}, one gets :

	\begin{eqnarray*}
		\bigg[~p'v~\bigg]_0^L & = & \cancel{p'(L)v(L)} - p'(0)v(0)\\
							  & = & - p'(0)v(0)
	\end{eqnarray*}

	Therefore, we have :

	\begin{eqnarray}
		-\int_0^Lp'v'\dx + k^2\int_0^Lpv\dx = p'(0)v(0), && \forall v \label{FEM1DPW:FV}
	\end{eqnarray}




\end{document}

