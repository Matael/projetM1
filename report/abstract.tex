\section*{Abstract}

As comptuation power and problems complexity increase, designing efficient numerical methods is a major challenge.

Some, among the large amount of available methods, have been known and used for years, but each is often limited to
certain types of problems. The key idea behind the present work is to prepare coupling between Finite Elements
Method (FEM) and Discontinuous Galerkin Method (DGM).

As FEM is particularly well-suited for small and detailled configurations and plane waves based DGM allows efficient modelling of
propagation in large cavities without too much details, coupling of both would lead to great improvements in
numerical computation of propagation in complex medium.

This document uses classical FEM formulation and the plane waves based DGM formulation recently proposed by
Gabard \textit{et al.}~\cite{Gabard15,Gabard11} as a basis. Both methods are succintly introduced and a coupling possibility
based on the rewriting of FEM boundary conditions using a DGM-compatible approach is then studied.

As this modification has a non-negligible impact on convergence of the method, a new interpolation functions basis
using Hermite splines is proposed.

The convergence rate and overall accuracy of the final method (based on rewritten boundary conditions and Hermite
splines) are finally shown to be better than those of the method based on classical FEM boundary conditions and
quadratic elements.

\paragraph{Keywords:} Helmholtz; finite elements; discontinuous Galerkin; coupling; convergence
