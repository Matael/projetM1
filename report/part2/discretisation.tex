\section{Discrétisation des champs}

Le choix des fonctions test est un enjeu prépondérant dans la méthode de Galerkin discontinue.
La suite cherche à discrétiser les champs de tests \textit{via} une base d'ondes planes, comme proposé par Gabard
\textit{et al.}\cite{Gabard15}.

Les expressions des champs discrétisés sont tirées directement de la résolution des équations
différentielles~\eqref{dgm:eq_u} et~\eqref{eq_v}. Il faut noter que le problème à résoudre pour $\ul{v}$ est l'adjoint
de celui à résoudre pour $\ul{u}$ (la matrice $\uul{A}$ n'étant pas symétrique).

Les solutions recherchés sont donc composées d'une somme d'ondes planes :

\[
    \xi(x) = \xi_+e^{-jkx}+\xi_-e^{+jkx}
\]

La solution pour $\ul{u}$ est donc de la forme :

\begin{equation}
    \ul{u} = \begin{Bmatrix}v\\p\end{Bmatrix} =
    \begin{bmatrix}
        \nicefrac{1}{Z_0} & -\nicefrac{1}{Z_0}\\
        1 & 1
    \end{bmatrix}
    \begin{bmatrix}
        e^{-jkx} & 0\\
        0 & e^{+jkx}
    \end{bmatrix}
    \begin{Bmatrix}
        A\\B
    \end{Bmatrix}
    =
    \underbrace{\begin{bmatrix}
        \nicefrac{e^{-jkx}}{Z_0} & -\nicefrac{e^{+jkx}}{Z_0}\\
        e^{-jkx} & e^{+jkx}
    \end{bmatrix}}_{\uul{U}_u(x)}
    \underbrace{\begin{Bmatrix}
        A\\B
    \end{Bmatrix}}_{\GU}
\end{equation}

Le champ de test $\ul{v}$ est solution du problème adjoint, la solution pour $\ul{v}$ est donc de la forme

\begin{equation}
    \ul{v} =
    \begin{bmatrix}
        Z_0 & -Z_0\\
        1 & 1
    \end{bmatrix}
    \begin{bmatrix}
        e^{-jkx} & 0\\
        0 & e^{+jkx}
    \end{bmatrix}
    \begin{Bmatrix}
        \delta A\\\delta B
    \end{Bmatrix}
    =
    \underbrace{
    \begin{bmatrix}
        Z_0e^{-jkx} & -Z_0e^{+jkx}\\
        e^{-jkx} & e^{+jkx}
    \end{bmatrix}}_{\uul{U}_v(x)}
    \underbrace{\begin{Bmatrix}
        \delta A\\\delta B
    \end{Bmatrix}}_{\GV}
\end{equation}
