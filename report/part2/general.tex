Les méthodes de Galerkin, découvertes dans les années 1910, permettent de chercher des solutions à des équations
différentielles en utilisant des hypothèses sur la forme de solution. Plus tard, dans les années 1970, la méthode de
Galerkin discontinue fut introduite pour résoudre des équations aux dérivées partielles. Enfin, très récemment,
l'utilisation des caractéristiques de l'EDP comme base pour la formulation des champs (test et solution) a été
introduite~\cite{Gabard11}.

\section{Généralités}

Soit un milieu de propagation où sont valable les équations suivantes :

\begin{eqnarray}
    \left\{\begin{array}{l}
        j\omega\rho v = -\nabla p\\
        p = -\rho c^2 \nabla u
    \end{array}\right.
    & \Leftrightarrow &
    \left\{\begin{array}{l}
        j\omega\rho v = -\nabla p\\
        j\omega p = -\rho c^2 \nabla v
    \end{array}\right.
    \notag\\ & \Leftrightarrow &
    j\omega \begin{Bmatrix}v\\p\end{Bmatrix} + 
    \begin{bmatrix}
        0 & \nicefrac{1}{\rho}\\
        \rho c^2 & 0
    \end{bmatrix}
    \nabla \begin{Bmatrix}v\\p\end{Bmatrix} = 0
    \notag\\ & \Leftrightarrow &
        \left(j\omega + \uul{A}\frac{\partial}{\partial x}\right) \ul{u} = 0 \label{dgm:eq_u}
\end{eqnarray}

L'idée est alors de découpler les équations précédentes en diagonalisant $\uul{A}$, il vient :

\begin{equation*}
    \uul{A} = \uul{P\Lambda Q} \quad,\quad \uul{P} = \uul{Q}^{-1}
\end{equation*}

En posant

\begin{equation}
    \ul{\tilde{u}} = \begin{Bmatrix}\tilde{u}^+\\\tilde{u}^-\end{Bmatrix} = \ul{Q}\ul{u}\label{dgm:utilde}
\end{equation}
l'équation~\eqref{dgm:eq_u} devient :

\begin{equation*}
    j\omega \ul{\tilde{u}} + \uul{\Lambda}\nabla\ul{\tilde{u}} = 0
\end{equation*}

En isolant les valeurs positives et négatives de $\uul{\Lambda}$  (notées $\Lambda^{+,-}$) ainsi que les vecteurs
associés $\ul{P}^{+,-}$ et $\ul{Q}^{+,-}$, il est possible d'écrire :

\begin{equation}
    \left\{\begin{array}{l}
        \tilde{u}^+ = \ul{Q}^+\ul{u}\\
        \tilde{u}^- = \ul{Q}^-\ul{u}
    \end{array}\right. \Leftrightarrow
    \ul{u} = \ul{P}^+\tilde{u} + \ul{P}^-\tilde{u} \label{dgm:sep_u}
\end{equation}


\paragraph{Formulation variationnelle}

En utilisant la formulation variationnelle et une intégration par parties, il vient :

\begin{equation}
    \begin{array}{c}
    \int_\Omega \ul{v}^T\left(j\omega + \uul{F}\nabla\right)\ul{u}\dd\Omega = 0 \quad,\quad \forall\ul{v}\\
    -\left(\int_\Omega j\omega v + \uul{A}^T\nabla\ul{v}\right)^Tu\dd\Omega +
        \int_{\partial\Omega}\ul{v}^T\uul{A}\ul{u}\dd\Gamma = 0 \quad,\quad \forall\ul{v}
    \end{array}
    \label{dgm:post_ipp}
\end{equation}

\paragraph{Nota Bene} Le symbole $\cdot^T$ représente une transposition Hermitienne.

L'objectif est alors de choisir le champ de test $\ul{v}$ permettant d'annuler l'intégrale sur $\Omega$ :

\begin{equation}
    \left(j\omega + \uul{A}^T\nabla\right)\ul{v} = 0 \label{dgm:eq_v}
\end{equation}

De toute l'équation~\eqref{dgm:post_ipp}, il reste alors :

\begin{equation}
    \int_{\partial\Omega}\ul{v}^T\uul{A}\ul{u}\dd\Gamma = 0 \quad,\quad \forall\ul{v}\label{dgm:to_solve}
\end{equation}

