Comme dit à la fin de la partie~\ref{part:couplage}, la formulation des conditions limites au moyen des
caractéristiques conduit à la dérivation des fonctions de forme classiques. Ce point est clairement un désavantage pour
cette méthode car elle force la perte d'un ordre de convergence.

Dans cette partie, un nouveau jeu de fonctions d'interpolation est proposé dans le but de pallier ce problème. Afin
d'évaluer la pertinence de la proposition, la comparaison est réalisée entre des éléments quadratiques et la base
proposée pour une formulation par caractéristiques d'une part et sans d'autre part.

\section{Une nouvelle interpolation : les splines d'Hermite}

Un moyen d'éviter la perte d'un ordre de convergence à l'utilisation de la méthode par caractéristiques serait
d'utiliser un ensemble de fonctions d'interpolation permettant l'accès direct à la dérivée de la quantité.

L'interpolation par splines d'Hermite utilise 4 «degrés de libertés» répartis sur les deux extrémité du segment à
interpoler : 2 pour le champ lui-même et 2 pour sa dérivée.

Le champ interpolé pour un segment $e$ d'extrémités $1$ et $2$ est défini tel que :

\begin{equation*}
	p_e(x) = \underbrace{\left[\H{00}(x)\big|\H{10}(x)\big|\H{01}(x)\big|\H{11}(x)\right]}_{H}\begin{Bmatrix}p_1\\p'_1\\p_2\\p'_2\end{Bmatrix}
\end{equation*}

Il en va de même pour $v_e$ en remplaçant simplement les valeurs dans le vecteur colonne et $p'_e$ ou $v'_e$ en
reprenant le vecteur colonne de leur homologue non dérivé et en dérivant les fonctions d'interpolation\footnote{Le
vecteur est alors noté $H'$.}.

Les expressions desdites fonctions sont explicitées en annexe~\ref{app:splines}.

En suivant une méthode tout à fait analogue à celle développée pour le calcul des matrices élémentaires en partie~\ref{part:fem}, il vient :

\begin{equation*}
	\uul{M_e} = H^TH \quad,\quad \uul{K_e} = H'^TH'
\end{equation*}

\subsection{Formulation classique}

La formulation classique revient, une fois de plus, à considérer le coefficient de réflexion $R$ comme une inconnue.

Cette méthode ayant déjà été détaillée en partie~\ref{part:fem}, elle ne sera pas ré-expliquée ici. Le fait de
considérer une interpolation par spines d'Hermite change en effet l'implémentation mais pas les expresions
fonctionnelles.

\subsection{Méthode des caractéristiques}

De manière tout à fait semblable à ce qui a été mis en place en partie~\ref{part:couplage}, il vient :

\begin{eqnarray*}
	p_e(0) & = & \underbrace{\left[\H{00}(0)\big|\H{10}(0)\big|\H{01}(0)\big|\H{11}(0)\right]}_{H}\begin{Bmatrix}\GP_1\\\GP_2\\\GP_3\\\GP_4\end{Bmatrix}\\
	v_e(0) = -\frac{1}{j\omega\rho}\nabla p\bigg|_0& = & -\frac{1}{j\omega\rho}\underbrace{\left[\Hp{00}(0)\big|\Hp{10}(0)\big|\Hp{01}(0)\big|\Hp{11}(0)\right]}_{H}\begin{Bmatrix}\GP_1\\\GP_2\\\GP_3\\\GP_4\end{Bmatrix}
\end{eqnarray*}

Le seul point important étant ici de bien garder en tête que le vecteur $\GP$ contient les valeurs du champ de pression
aux points voulues entrelacées avec les valeurs de la dérivée de ce même champ.

En évaluant les polynômes et leurs dérivées en $0$, et avec $\tilde{\ul{u}}_0 = \uul{Q}\ul{u}_0$ :

\begin{eqnarray*}
\begin{Bmatrix}\tilde{u}^+\\\tilde{u}^-\end{Bmatrix}_0 & = &
\begin{bmatrix}
	\nicefrac{1}{2} & \nicefrac{1}{2Z_0}\\
	\nicefrac{1}{2} & -\nicefrac{1}{2Z_0}\\
\end{bmatrix}
\begin{bmatrix}
	0 & -\frac{1}{j\omega\rho} & 0 & 0\\
	1 & 0 & 0 & 0\\
\end{bmatrix}
\begin{Bmatrix}\GP_1\\\GP_2\\\GP_3\\\GP_4\end{Bmatrix}\\
& = & \begin{bmatrix}
	\nicefrac{1}{2Z_0} & -\frac{1}{2j\omega\rho} & 0 & 0\\
	-\nicefrac{1}{2Z_0} & -\frac{1}{2j\omega\rho} & 0 & 0\\
\end{bmatrix}
\begin{Bmatrix}\GP_1\\\GP_2\\\GP_3\\\GP_4\end{Bmatrix}\\
\end{eqnarray*}

Sachant que~\eqref{rflx:useful_rels} et~\eqref{rflx:def_v} sont toujours valables, il vient :

\begin{eqnarray*}
v(0) & = & \tilde{u}^+_0+\tilde{u}^-_0 = \frac{1}{Z_0} - \frac{1}{2Z_0}\GP_1 - \frac{1}{j\omega\rho}\GP_2\\
\nabla p\bigg|_0 & = & -j\omega\rho v(0) = -jk + \frac{jk}{2}\GP_1 + \frac{1}{2}\GP_2
\end{eqnarray*}

Cette expression est alors à remplacer dans le second membre de~\eqref{FEM1D:post_BCL} et la résolution peut se faire
par simple division.
