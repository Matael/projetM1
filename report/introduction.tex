\section*{Introduction}

À mesure que la puissance des ordinateurs explose et que la science avance, les méthodes numériques représentent un
enjeu de plus en plus important. Aujourd'hui, la complexité des problèmes engagés est telle que les solutions
analytiques sont parfois impossibles à déterminer.

Après des années de recherche, de plus en plus de méthodes ont vu le jour pour pallier ce manque et ce dans quasiment
tous les domaines scientifiques. Depuis les méthodes spectrales jusqu'aux algorithmes évolutifs, la science moderne est
avide de méthodes d'approximation et d'automatisation efficaces et celles-ci sont en constant développement. De plus,
la comparaison entre les différentes méthodes présente un intérêt majeur pour réaliser ensuite des choix
avisés~\cite{Gabard11}.

Connues depuis des décennies, les méthodes visant à discrétiser le milieu de travail ou le temps sont un des cas
d'étude classiques, la méthode des éléments finis (FEM) en fait partie. Pourquoi ne pas résoudre plein de petits problèmes
simples plutôt qu'un unique problème complexe ? La discrétisation des variables n'est toutefois pas l'unique solution
envisageable...

Mises au point au début du XX\textsuperscript{ème} siècle, les méthodes de Galerkin suivent, elles, un autre chemin vers le
problème discret : en plus d'une discrétisation en espace, elles proposent aussi de chercher la solution du problème
sous la forme d'une somme de fonctions plus simples. Améliorées dans les années 1970 en méthodes de Galerkin
discontinues (DGM), elles permettent alors la résolution d'équations aux dérivées partielles (EDP).

Ces deux méthodes sont classiques et bien connues des chercheurs, elles sont toutefois leurs spécificités quant aux
problèmes qu'elles permettent de résoudre de manière consistante.

\bigskip

L'objectif de ce projet est de s'intéresser au couplage entre la FEM et la DGM en se basant sur la ré-écriture des
conditions limites de la FEM.

Dans un premier temps, les deux méthodes sont succintement présentées et un exemple d'utilisation est proposé.
Dans un second temps, la possibilité de ré-écriture des conditions aux limites du domaine FEM est analysée et une
nouvelle formulation basée sur le parallèlisme avec les caractéristiques utilisées dans la DGM avec ondes planes est
proposée. Une réflexion sur la convergence de cette méthode modifiée amène ensuite à considérer la dernière partie de ce
rapport. Afin de rendre la méthode proposée réellement intéressante, une nouvelle base de fonctions d'interpolation pour
les champs FEM est introduite et son influence sur la convergence est analysée.



