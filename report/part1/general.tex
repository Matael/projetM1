Parmi les méthodes de discrétisation les plus souvent citées, la méthode des éléments finis tient certainement une
des meilleurs places. Dans cette partie, la méthode est succintement présentée ainsi que les fonctions d'interpolation
courantes. Enfin, la simulation d'une cavité acoustique 1D est implémentée avec un double objectif : discuter de la
prise en compte des conditions limites en éléments finis d'une part et introduire la notion de convergence et son
enjeu d'autre part.

\section{Généralités}
\label{FEM1D:subsection:linear}

\subsection{Formulation Variationnelle}

Le problème est régi par l'équation d'Helmholtz sans source telle que présentée en~\eqref{FEM1D:helm} (avec $k =
\nicefrac{\omega}{c}$ le nombre d'onde, $c$ la célérité du son dans le milieu, et $\omega$ la pulsation).

\begin{equation}
	(\Delta + k^2)p(x,\omega) = 0 \label{FEM1D:helm}
\end{equation}

En faisant usage de la formulation variationnelle (avec $\DP$ le champ variationnel), puis d'une intégration par
parties, le problème s'exprime comme en~\eqref{FEM1D:IPP}.

\begin{eqnarray}
	\int_\Omega \Delta p~\DP~\dd\Omega + k^2 \int_\Omega p~\DP~\dd\Omega & =  & 0,\quad \forall\DP\notag\\
	- \int_\Omega \nabla p~\nabla \DP ~\dd\Omega + k^2\int_\Omega p~\DP~\dd\Omega & = & -\int_{\partial\Omega}\nabla p~\DP\dd\Gamma, \quad \forall\DP \label{FEM1D:IPP}
\end{eqnarray}

\subsection{Fonctions de forme}

Supposant que les champs $p$ et $\DP$ sont décomposables sur un intervalle $[x_i, x_j]$ en une combinaison linéaire de
fonctions $\phi_k$ (dites fonctions de forme) assorties des valeurs du champ à des points particuliers $x_k$ du domaine, il
vient (cas 1D):

\begin{equation}
	\left\{\begin{array}{l}
		p = \sum_{k=1}^N \phi_k(x)p(x_k)\\
		p' = \sum_{k=1}^N \phi_k'(x)p(x_k)\\
	\end{array}\right., \quad
	\left\{\begin{array}{l}
		\DP = \sum_{k=1}^N \phi_k(x)\DP(x_k)\\
		\DP' = \sum_{k=1}^N \phi_k'(x)\DP(x_k)\\
	\end{array}\right.\label{FEM1D:shapefun}
\end{equation}

Pour alléger les notations, on pose $\alpha_k\equiv \alpha(x_k)$, où $\alpha$ est un champ et $x_k$ un point.

Il est possible de ré-écrire les équations de~\eqref{FEM1D:shapefun} sous forme vectorielle comme présenté
en~\eqref{FEM1D:shapevec}.

\begin{equation}
        p = \big[\phi_1\big|\cdots\big|\phi_N\big]\begin{Bmatrix}p_i\\\vdots\\p_j\end{Bmatrix} =
			\big[\phi_1\big|\cdots\big|\phi_N\big]\ul{p}\label{FEM1D:shapevec}
\end{equation}

Et de même pour les champs $p'$, $\DP$ et $\DP'$. On note dans la suite $\ul{\phi}$ et $\ul{\phi'}$ les vecteurs des fonctions de
forme et de leurs dérivées respectivement.

En considérant une partition du domaine $\Omega$ telle que $\Omega = \sum_e\Omega_e$ où~\eqref{FEM1D:helm} est vérifiée
sur chacune $\Omega_e$ et en utilisant cette dernière équation~\eqref{FEM1D:shapevec} --- cf. ~\eqref{FEM1D:IPP} :

\begin{eqnarray*}
	- \int_\Omega \nabla p~\nabla \DP ~\dd\Omega + k^2\int_\Omega p~\DP~\dd\Omega & = & -\int_{\partial\Omega}\nabla p~\DP~\dd\Gamma, \quad \forall\DP\\
\Leftrightarrow \sum_e\left\{- \int_{\Omega_e} \nabla p_e~\nabla \DP_e ~\dd\Omega + k^2\int_{\Omega_e} p_e~\DP_e~\dd\Omega \right\}& = & -\int_{\partial\Omega}\nabla p~\DP~\dd\Gamma, \quad \forall\DP\\
	\Leftrightarrow \sum_e\left\{- \int_{\Omega_e} \ul{\DP}_e^T\ul{\phi'^T\phi'}\ul{p}_e~\dd\Omega + k^2\int_{\Omega_e} \ul{\DP}_e^T\ul{\phi^T\phi}\ul{p}_e~\dd\Omega \right\}& = & -\int_{\partial\Omega}\nabla p~\DP~\dd\Gamma, \quad \forall\DP\\
\end{eqnarray*}


Dans les équations précédentes, les quantités indicées d'un $e$ sont valables sur un élément. Les quantités soulignées
une fois sont des vecteurs, celles soulignés deux fois des matrices\footnote{Le nombre de soulignements correspondant à
l'ordre tensoriel de la quantité.}.
Après calcul des produits $\ul{\phi^T\phi}$ et $\ul{\phi'^T\phi'}$ et passage des intégrales dans les matrices ainsi
obtenues, il vient :

\begin{equation}
	\sum_e\left\{- \ul{\DP^T}~\uul{K}_e~\ul{p} + k^2 \ul{\DP^T}~\uul{M}_e~\ul{p} \right\} = -\int_{\partial\Omega}\nabla p~\DP\dd\Gamma, \quad \forall\DP\label{FEM1D:elem_mat}\\
\end{equation}

avec :

\begin{equation*}
	\uul{M}_e = \begin{bmatrix}
		\int_{\Omega_e}\phi_1^2\dd\Omega & \int_{\Omega_e}\phi_1\phi_2\dd\Omega & \cdots  &\int_{\Omega_e}\phi_1\phi_N\dd\Omega\\
		\int_{\Omega_e}\phi_2\phi_1\dd\Omega & \int_{\Omega_e}\phi_2^2\dd\Omega & &\\
		\vdots & & \ddots & \vdots\\
		\int_{\Omega_e}\phi_N\phi_1\dd\Omega & & \cdots & \int_{\Omega_e}\phi_N^2\dd\Omega
	\end{bmatrix}
\end{equation*}

La matrice $\uul{K}_e$ est obtenue en considérant les dérivées $(\phi'_1,\ldots,\phi'_N)$ des fonctions de forme plutôt que les
fonctions elles-mêmes. Ces matrices sont appelées matrices élémentaires.

\subsection{Matrices Booléennes et Vecteurs Globaux}
\newcommand{\GP}{\ul{\mathbb{P}}}
\newcommand{\GDP}{\ul{\delta\mathbb{P}}}
L'objectif est maitenant d'exprimer l'intérieur de la somme non plus en fonction des extrémités de l'éléments en cours
$\ul{p}_e$ et $\ul{\DP}_e$ mais des vecteurs globaux $\GP$ et $\GDP$.

En remarquant que $\ul{p}_e$ sur le premier élément s'exprime en fonction de $\GP$ \textit{via} :

\begin{equation*}
    \ul{p}_e =
    \underbrace{\begin{bmatrix}
         1 & 0 & 0 & \cdots & 0\\
         0 & 1 & 0 & \cdots & 0
    \end{bmatrix}}_{\uul{L}_e}\GP
\end{equation*}

De même, $\ul{\DP}_e$ d'obtient \textit{via} $\ul{\DP}_e^T = \GDP^T\uul{L}_e^T$. En remplaçant dans~\eqref{FEM1D:elem_mat}, il
vient :

\begin{equation}
    \sum_e \left\{
	\GDP^T\uul{L}_e^T\uul{K_e}\uul{L}_e\GP~\dd\Omega - k^2 \GDP^T\uul{L}_e^T\uul{M_e}\uul{L}_e\GP~\dd\Omega \right\}
	= -\int_{\partial\Omega}\nabla p~\DP\dd\Gamma, \quad \forall\GDP\label{FEM1D:post_boolean}
\end{equation}


Si on considère un problème 1D de longueur L (tel que présenté en figure~\ref{fig:FEM:propa_1D}), alors le second membre
peut s'écrire :

$$-\int_{\partial\Omega}\nabla p~\DP\dd\Gamma = -\left(\nabla p\bigg|_L\GDP_N- \nabla p\bigg|_0\GDP_0\right)$$

Comme ni $\GP$ ni $\GDP$ ne dépendent de l'élément, il est possible de les sortir de la somme, qui est ensuite
distribuée sur les deux termes comme suit :

\begin{equation*}
	\GDP^T\left[\underbrace{\sum_e \left\{ \uul{L}_e^T\uul{K_e}\uul{L}_e\right\}}_{\uul{K}}
	- k^2 \underbrace{\sum_e\left\{\uul{L}_e^T\uul{M_e}\uul{L}_e\right\}}_{\uul{M}}\right]\GP
	= -\left(\nabla p\bigg|_L\GDP_N- \nabla p\bigg|_0\GDP_0\right), \quad \forall\GDP
\end{equation*}

Le fait que cette expression soit valable pour tout $\GDP$, implique :

\begin{equation}
\left[- \uul{K} + k^2\uul{M}\right]\GP = \begin{Bmatrix} \nabla p\bigg|_0\\0\\\vdots\\0\\-\nabla p\bigg|_L\end{Bmatrix} \label{FEM1D:final}
\end{equation}
