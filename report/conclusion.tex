\section*{Conclusion}

La présentation et l'introduction aux méthodes considérées, qui devait être accessoire initialement, est finalement
devenu le point principal de ce projet. Les méthodes étant nouvelles il aura fallu les prendre en main et apprendre
à s'en servir, ce qui n'était pas forcément aisé.

Le second temps, à savoir le couplage entre ces deux méthodes, s'est avéré au moins aussi instructif que le premier et a
conduit à une réflexion autour des effets liés au couplage numérique. Plus qu'écrire des équations, cela aura demandé
d'asseoir les connaissances sur les méthodes en jeu et de prendre en compte des petits détails qu'il est souvent
agréable de passer sous silence.

Cette conclusion marque ainsi la fin d'un projet extrêmement instructif et intéressant mais aussi --- et surtout --- le début d'une
suite. En effet, beaucoup de pistes sont envisageables pour poursuivre la quête du couplage idéal  : aucune vraie
expérience n'a été menée dans ce projet qui se contente de poser les bases du concept. De plus, tous les cas
présentés ici sont des cas 1D, le passage en 2D, s'il est avant tout technique, demandera très certainement un travail
supplémentaire pour écrire les équations du couplage. Une analyse de la rapidité d'une méthode mixte aurait aussi un
intérêt non négligeable et passera probablement par la mise en place de tests objectifs ; suivant les résultats,
certains pans des scripts devront peut être être réécrits pour accélérer le tout. Pour finir sur une touche plus
informatique, l'analyse des maillages aura aussi une importance capitale pour choisir automatiquement la méthode la
plus adaptée à chaque partie du maillage.

Au fur et à mesure des années, je tends à me rapprocher du monde des numériciens et ce projet m'a beaucoup plu pour ça.
Les discussions avec Olivier Dazel concernant les mathématiques et l'analyse numérique m'ont fait entrevoir qu'il y
avait, derrière une première couche un peu difficile d'accès, un monde d'ingéniosité pour contourner de manière élégante
les limites de la science. Ce projet a définitivement eu l'avantage de me forcer à creuser plus avant des points que je
pensais commencer à maîtriser : la remise en question est toujours un bon moteur.
